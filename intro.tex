\intro{}
Інтелектуальна обробка інформації вже дуже давно зарекомендувала себе в сучасному світі.
Людство генерує незчисленні обсяги інформації щосекунди.
Багато згенерованої інформації (поки що) не зберігається, а з того, що зберігається, багато ніяк не аналізується.

Але з поширенням інформаційних технологій та з розповсюдженням пристроїв, що можуть ефективно обробляти великі обсяги
інформації, знаходиться все більше сфер, де можливо отримати користь з раніш не оброблюваного інформаційного потоку.

Наприклад, аналізуючи зміни погодних умов упродовж дня, метеорологи мають змогу прогнозувати погоду наперед;
аналізуючи різні показники життєдіяльності людини, лікарі знаходять прояви хвороб ще до перших її симптомів.

Задачі пошуку окремих наперед визначених властивостей чи образів у деякому масиві даних набули великого поширення в
галузях, пов’язаних з обробкою цифрової інформації.

Співставлення з еталоном (template matching) широко застосовується в аналізі як одновимірних, так і двовимірних
сигналів~\cite{book4}.

Зазвичай, постановка такої задачі включає в себе дані для аналізу, деякий шаблон для пошуку та, можливо, область, на
якій потрібно здійснювати такий пошук.  При цьому потрібно як правильно обрати представлення вхідних даних, так і
визначити критерії, за якими відбуватиметься зіставлення шаблону з сигналом~\cite{book10}.

Існує розмаїття підходів та метрик, які можна використовувати для вирішення цієї задачі.

У даній роботі для цього пропонується використовувати поліноми Кунченка.

% vim: spelllang=uk,en spell
