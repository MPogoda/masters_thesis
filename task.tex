\chapter{Постановка задачі}

В магістерський дисертації ставиться задача пошуку шаблонів у цифрових сигналах за допомогою поліномів Кунченка.
Дана задача включає наступні задачі:

\begin{enumerate}[label=\arabic*. ]
    \item Проаналізувати існуючі методи для пошуку шаблонів у цифрових сигналах і обґрунтувати вибір методу пошуку
        шаблонів на основі поліномів Кунченка.
    \item Розробити алгоритм пошуку шаблонів на основі поліномів Кунченка.
    \item Проаналізувати шляхи прискорення роботи та покращення результатів роботи розробленого алгоритму.
    \item Дослідити методи з обробки цифрових сигналів, зокрема аудіосигналів, що містять мовлення.
    \item Розробити програмне забезпечення для пошуку шаблонів у цифровому сигналі.
    \item Проаналізувати вплив зміни параметрів алгоритму на швидкість та ефективність пошуку шаблонів.
    \item Провести статистичний експеримент для порівняння ефективності роботи розробленого програмного забезпечення з
        існуючими методами.
\end{enumerate}

% vim: spelllang=uk,en spell
