\chapter{Метод Кунченка}

Відомо, що поліном Тейлора є одним з найкращих способів апроксимації функції на деякому інтервалі, проте задля
використання цього методу необхідно забезпечити $n$-раз гладкість функції.

Кунченком~Ю.\,П. було запропоновано використовувати для апроксимації поліном, побудований з неортонормованого базису
функцій, що були отримані шляхом певних перетворень з шаблона.

В цьому розділі буде запропонований метод пошуку шаблонів за допомогою апроксимації поліномів у просторі Кунченко.
Для цього буде розглянутий лінійній простір Кунченко, метод розкладання довільної функції в цьому просторі й
застосування цього розкладення в проблемі пошуку шаблонів.
Також будуть розглянуті методи, за допомогою яких можна поліпшити результати алгоритму.

\section{Лінійній простір Кунченка}
    Нехай $f(x)$ --- так звана породжуюча функція, що визначена на проміжку ${[a, d]}$, а $\{ \phi_v( x ) \}$ ---
    довільна множина функціональний перетворень.
    Тоді, можна породити множину функцій
    \begin{equation}
        u_v(x) = \phi_v( f( x ))
    \end{equation}

    З цієї множини можна вибрати підмножину лінійно"=незалежних функцій.
    Лінійний простір, утворений на такій множині, називатимемо лінійним простором над незалежними породженими
    функціями, або лінійний простором Кунченка (LFKu).

    На цьому просторі можна звичайним чином визначити скалярний добуток:
    \begin{equation}
        \Psi_{v,k} = (u_v( x ), u_k( x ) ) = \int\limits_a^d u_v( x ) u_k( x )\,\mathrm{d}x,
    \end{equation}
    та відстань між двома елементами:
    \begin{equation}
        \rho_{v,k}^2 = \|u_v( x ) - u_k( x )\|^2 = \int\limits_a^d\left(u_v(x) - u_k(x)\right)^2\,\mathrm{d}x
    \end{equation}

    Виберемо деяку породжену функцію $u_b(x)$, яку будемо називати основною.
    Тоді множину функцій $u_v(x), v \ne b$ будемо називати доповнювальними функціями.

    Можна утворити поліном Кунченка з доповнювальних функцій:
    \begin{equation}
        \label{eq:polynom}
        P_t( x ) = \sum^t_{k = 1, k \ne b} a_k u_k( x  )
    \end{equation}

    Відстань між основною функцією $u_b(x)$ та утвореним поліномом за визначенням буде дорівнювати:
    \begin{equation}
        \rho^2_{b,P} = \int\limits_a^d\left( u_b(x) - P_t(x)\right)^2 \, \mathrm{d}x
    \end{equation}

    Можна показати, що коефіцієнти $a_k, k \ne 0$ знаходяться з системи лінійних рівнянь:

    \begin{equation}
        \label{eq:linear-system}
        \sum^t_{k=1,k\ne b} a_k F_{v,k} = F_{v,b},\qquad v=\overline{0,t}, v \ne b,
    \end{equation}
    де
    \begin{eqnarray}
        \label{eq:centered-correlants}
        F_{v,k} &\equiv& \Psi_{v,k} - \Psi_v \Psi_k \| u_0( x ) \|^2\\
        \Psi_v &\equiv& \frac{\Psi_{v,0}}{\left\|u_0(x)\right\|^2}\\
        a_0 &=& \Psi_b - \sum\limits_{v=1, v \ne b}^r{a_v \Psi_v}
    \end{eqnarray}

    Спростив вираз, отримаємо:
    \begin{equation}
        \rho^2_{bP} = F_{b,b} - J_t,
    \end{equation}
    де $J_t$ --- інфоркуна полінома, що має вигляд:
    \begin{equation}
        J_t \equiv \sum_{v=1,v\ne b}^t a_v F_{v,b}
    \end{equation}

    Тоді мірою апроксимації поліномом будемо вважати наступну величину:
    \begin{equation}
        \label{eq:et}
        e_t = \frac{J_t}{\int\limits_a^d (u_b(x)-\Psi_b u_0( x ))^2\, \mathrm{d}x}
    \end{equation}

\section{Застосування розкладення для пошуку шаблону в сигналі}
    Візьмемо в якості породжуючої функції шаблон $f(x)$, що є визначеним на проміжку $\forall x_i \in {[a, d]}$.
    Тоді, нехай $\{u_v(x)\}$, також визначені $\forall x_i \in {[a, d]}$ --- породжені функції від шаблону, що
    утворюють лінійній простір Кунченко.

    Якщо взяти за основну функцію $u_b(x)$ вхідний сигнал на проміжку ${[a,d]}$, за допомогою~\eqref{eq:polynom} можна
    побудувати поліном наближення сигналу $P_t(x)$.

    Оскільки поліном, визначений таким чином буде являти собою наближення вхідного сигналу за допомогою певних
    перетворень з шаблону, то можна вважати величину~\eqref{eq:et} мірою схожості вхідного сигналу до шаблону на
    обраному проміжку.

    Таким чином, скомбінував метод ковзаючого вікна (розділ~\ref{ss:sliding-window}) з розкладенням сигналу в
    лінійному просторі Кунченко, можна отримати ефектограму, тобто функцію $e_t(x)$, що буде характеризувати схожість
    вхідного сигналу зі шаблоном у певному вікні.

    Оскільки ефектограма може приймати значення від $-1$ до $1$, можливо визначити поріг $e_0$, такий, що значення
    більше (або менше при умові його від’ємності) буде характеризувати відповідне вікно як таке, що містить
    модифікований шаблон.

    Цей алгоритм потребує наступні обчислення:
    \begin{itemize}
        \item $O(t^2) O(k^2)$ обчислень для підрахунку попарних центрованих корелянт~\eqref{eq:centered-correlants}
            всіх породжених функцій (кожна з яких має $k$ значень).
            Ця дія має бути виконаною лише один раз на початку пошуку, оскільки при ковзанні вікна вони не будуть
            змінюватись.
        \item $O(t) O(k)$ обчислень для підрахунку корелянт із основною функцією (вхідним сигналом)~$F_{v,b}$ на
            кожному кроці, оскільки значення основної функції змінюється із зсувом вікна.
        \item $O(t^3)$ обчислень для пошуку коренів системи лінійних рівнянь~\eqref{eq:linear-system}.
        \item $O(t) + O(k^2)$ для обрахування ефективності наближення на кожному кроці.
    \end{itemize}

\section{Методи покращення пошуку шаблонів}
    Як видно з попередніх обчислень, запропонований метод має більшу обчислювальну складність, ніж метод із
    використанням нормалізованої взаємокореляції (розділ~\ref{s:existing-compare}).

    В цьому підрозділі будуть запропоновані методи прискорення роботи алгоритму.

    \subsection{Пірамідальний пошук}
        Оскільки основною складовою кількості обчислень можна вважати $O(k^2)$, було запропоновано використання методу
        так званого пірамідального пошуку.
        \todo[inline]{Vanderbrug, G.J. and Rosenfeld, A. (1977) Two-stage Template Matching. IEEE Transactions on Computers, C-26,}
        \todo[inline]{Rosenfeld, A. and Vanderbrug, G.J. (1977) Coarse-Fine Template Matching. IEEE Transactions on Systems, Man and Cybernetics, 7, 104-107.}

        Ідея цього методу полягає в тому, що спочатку виконується певне масштабування (стиснення) шаблону та сигналу.
        Після цього виконується пошук стисненого шаблону в стисненому сигналі.
        Якщо знаходяться екстремуми, що характеризують знахідку шаблону, то на відповідній області нестисненого
        сигналу виконується пошук нестисненого шаблону.

        Таким чином, якщо сигнал та шаблон були промасштабовані із співвідношенням $1:10$, то головна складова
        кількості обчислень зміниться до $\frac{1}{100} O(k^2)$.
        Як правило, після першого етапу залишається значно менше областей, в яких потрібно уточнювати результат.

        Також варто зазначити, що можливі два підходи щодо стиснення сигналів:
        \begin{itemize}
            \item Зміна частоти дискретизації (resampling).
                В цьому випадку можна використовувати, наприклад фільтр Ланцоша, якщо орігінальна частота
                дискретизації не ділиться на нову.
                Проте значно ефективніше обирати таку нову частоту дискретизації, що ділить орігінальну частоту
                націло.
                В такому випадку можна вибрати кожне $m$-те значення сигналу та шаблону, де $m$ --- співвідношення
                частот.
                \todo[inline]{Ken Turkowski and Steve Gabriel (1990). "Filters for Common Resampling Tasks". In Andrew
                    S. Glassner. Graphics Gems I. Academic Press. pp. 147–165. ISBN 978-0-12-286165-9.}
            \item Брати середнє значення від деякого околу точки.
        \end{itemize}
    \subsection{Метод прискореного ковзання}
        Ідея методу полягає в динамічній зміні шагу ковзання в залежності від зміни $\Delta x_i = e_t(x_i) -
        e_t(x_{i-1})$.
        Наприклад, таким чином можна не аналізувати вікно, якщо ефективність наближення на попередньому вікні
        погіршилася.
        Але слід зазначати що через природу функції $e_t(x)$ не можна збільшувати шаг зсуву вікна нескінченно.
        Максимальне значення цього шагу потрібно задавати використовуючи певні евристики.

\section{Висновки}
В цьому розділі був визначений лінійній простір над породжуючою функцією, або лінійний простір Кунченко.
Було показано, як використовуючи наближення функції елементами базису цього простору можна вирішити задачу пошуку
шаблону в цифровому сигналі.
Обраний метод має більш нечітку оцінку присутності шаблону в сигналі, ніж розглянуті в попередньому розділі.
Також були розглянуті методики, завдяки яким можна прискорити роботу методу.

% vim: spelllang=uk,en spell filetype=tex
