\annotation{Реферат}
Метою цієї магістерської дисертації є розробка та оптимізація програмного комплексу для пошуку шаблонів у цифровому
сигналі.
Комплекс, що розробляється, призначений для виділення ділянок цифрового сигналу, що містять деякий шаблон, можливо у
модифікованому вигляді.

Розглянуто існуючі методи пошуку шаблонів у цифрових сигналах на прикладі нормалізованої кросс"=кореляції та суми
квадратів відстаней, запропонований новий метод і проаналізовані шляхи покращення його роботи.
Розроблений комплекс протестований на низці тестових даних, в тому числі на аудіосигналах з мовленням.

Робота складається з вступу, \total{chapter}  розділів та висновків і налічує \total{page} сторінок.
Містить \total{figurecount} ілюстративних матеріалів, \total{tablecount} таблиць, \total{appendnum} додатки та
посилається на \total{bibitemcount} літературних джерел.

Ключові слова: пошук шаблонів, цифровий сигнал, обробка сигналів, розпізнавання, математичний метод, поліноми
Кунченка.
\clearpage

\annotation{Abstract}
In this paper, the system for template matching in digital signal was developed and optimised.
Program was designed to correctly match subintervals of signal to some template that may be altered in some way.

Also described existing methods for template matching in digital signals such as normalised cross"=correlation and sum
of squared distances, new method was proposed.
Ways for optimisation for new method were observed.
Developed software was tested alongside with existing methods on speech signal.

The work consists of an introduction, \total{chapter} sections, includes conclusions and \total{page} pages.
Contains \total{figurecount} illustrative materials, \total{tablecount} tables, \total{appendnum} appendices and has
\total{bibitemcount} references.

Keywords: template matching, digital signal, signal processing, recognition, mathematical algorithm, Kunchenko
polynomials.

% vim: spelllang=uk,en spell
